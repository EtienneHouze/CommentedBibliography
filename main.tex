\documentclass{article}

\usepackage[utf8]{inputenc}
\usepackage[T1]{fontenc}

\title{Commented Bibliography}
\author{\'Etienne Houzé}
\date{}

\begin{document}
\maketitle
    \section{Read}

    \cite{marcus2018innateness} is a discussion on the article presenting AlphaZero as a \emph{tabula rasa} method \cite{silver2017mastering}. It objects that since it has been engineered by experts in the go game, its very architecture already carries some specificities aimed at solvig this particular problem.

    \cite{russell2016artificial} is a good handbook presenting the main techniques in AI. In particular, it describes knowledge-based systems and logic representation of the environment (Part III). Other parts are interesting too, but not as much linked to this work.

    \cite{dessalles2015conceptual}  presents an approach to define and name concepts in a geometric space (Contrast). It comes from the works in \cite{gardenfors2004conceptual} which introduces the existence of dimensions in the conceptualisation of the environment.

    \cite{dessalles2008computational} defines a conflict as a situation when an event happens but is not desired, or when a desired event is not observed. It also presents a conflict resolution based on CAN. This approach can be used to generate dialogue between the user and the system in case of a conflict.

    \cite{dovsilovic2018explainable} is a quick global survey of the state of explanation in AI today. It offers a succint view but states well the current struggles of technologies used in most state-of-the-art solutions.

    \cite{zimmermann2017user} is a short surver on the desires of users in Germany about the smart home. It concludes that, with Germany being the main market in Europe (\$1.3B), it still has room for increase (for instance, in the US, smart homes business is worth \$13B). It also shows on a very small sample that it seems users tend to put energy savings as the first incentive to use a smart home, with a predominance of independence for the older users (above 50).

    \section{To read}

    \cite{amarasinghe2018toward} examines a framework to explain decision made by DNNs in case an anomaly happens.
\newpage
\bibliographystyle{unsrt}
\bibliography{biblio}
\end{document}